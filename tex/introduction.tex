Word embeddings, which represent the semantic meaning of text data as vectors, are used as input in natural language processing tasks. It has been found that word embeddings exhibit unexpected social biases, such as gender bias, that are present in their training corpora \citep{bolukbasi2016man, caliskan2017semantics,garg2018word}. Contextual word embedding models, such as BERT \citep{devlin2018bert}, have become increasingly common and achieved new state-of-the-art results in the many NLP tasks. Researchers have also found gender bias in contextualized embeddings \citep{zhao2019gender,may2019measuring}.

A common approach for removing gender information in static embeddings is to identify a linear gender subspace (e.g., a gender direction) and subsequently setting all values on the gender direction to 0. Successful approaches rely on simple principal component analysis \cite{bolukbasi2016man,mu2018all}. \citet{bolukbasi2016man} requires pairs of gendered words to compute a direction (e.g., ``man''-``woman'') and \citet{mu2018all} relies on computing a PCA of a set of gender words hoping that the main variation occurs across gender. We propose to use DensRay \citep{dufter2019analytical}: the main advantage is that DensRay only requires two or multiple groups of gendered words. In contrast to \cite{bolukbasi2016man} it does not require explicit pairs and compared to \cite{mu2018all} it has explicit supervision with gender labels. We show in an artificially created example that DensRay is more stable. 

In summary our contributions are: i) We adjust DensRay to work on contextualized embeddings. 
 We apply DensRay to every BERT layer and evaluate two tasks: a set of templates we constructed and the Word Embedding Association Test (WEAT) \citep{caliskan2017semantics}. Our experiments find that debiasing with DensRay effectively mitigates gender bias and performs on par-with prior approaches. ii) We argue that DensRay is more robust and interpretable than prior approaches.
iii) We investigate whether debiased models maintain the performance of BERT on language modeling and the GLUE tasks \citep{wang2018glue}. iv) We apply our debiasing method to the multilingual-BERT (mBERT) model: we use English gender label for computing the gender subspace, and show that this effectively debiases Chinese. 

\enote{pd}{
I would suggest a different structure for the paper: 
Introduction
Methods
	Hard Debiasing
	DensRay (potentially add the robustness figure here)
	Debiasing Contextualized Embeddings
Experiments
	Setup and Data
	Evaluation
		Templates
		WEAT
		GLUE
Results
	Debiasing Results (OCCTMP, WEAT)
	Model Performance
	Examples
	Analyses (Layer, Attention Heads, Requried samples)
	Multilingual Debiasing
Related Work
Conclusion
}