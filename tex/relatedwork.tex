\subsection{Quantifying Gender Bias}
A typical way to measure gender bias is to evaluate on
\textbf{downstream tasks}. For coreference resolution,
\citet{zhao2018gender} designed Winobias and
\citet{rudinger2018gender} designed Winogender schemas. In
contrast to WinoBias, Winogender schemas include
gender-neutral pronouns. One Winogender schema has one
occupational mention and one ``other participant'' mention
while WinoBias has two occupational mentions. \enote{pd}{is
  the difference between Winobias and Winogender relevant to
  this work?}\citet{webster2018mind} released GAP, a
balanced corpus of Gendered Ambiguous Pronouns, which
measures gender bias as the ratio of F1 score on masculine
to F1 score on feminine. However the ratio is very close to 1 \citep{Chada_2019, Attree_2019} making it hard to compare debiasing systems. For sentiment analysis, Equity Evaluation Corpus (EEC) \citep{Kiritchenko_2018} was designed to measure gender bias by the difference in emotional intensity predictions between gender-swapped sentences.

An alternative way to measure gender bias is based on \textbf{association tests}, which originated from sociological research. \citet{greenwald1998measuring} proposed the Implicit Association Test (IAT) to quantify societal bias. In the IAT, response times were recorded when subjects were asked to match two concepts. For example, subjects were asked to match black and white names with ``pleasant'' and ``unpleasant'' words. Subjects tended to have shorter response times for concepts they thought associated. Based on the IAT, \citet{caliskan2017semantics} proposed the Word Embedding Association Test (WEAT), which uses word similarities between targets and attributes instead of the response times to get rid of the requirement of human subjects. \citet{may2019measuring} extended WEAT to the Sentence Embedding Association Test (SEAT); \citet{kurita2019measuring} proposed a template-based log probability bias score to measure the association between targets and attributes in BERT.

\enote{hs}{for many of the papers you discuss above it's not
  clear what the realtinship to the current work is. this
  hsould always be clear}

\subsubsection{Word Embedding Association Test}
\label{sec:weat}
Here we introduce WEAT in detail. Consider two sets of
target words $X_1,X_2$ with equal size $|X_1|=|X_2|$, and
two sets of attribute words $A_1,A_2$ with
$|A_1|=|A_2|$. The null hypothesis in the statistical test
of WEAT is: there is no difference in the cosine similarity
between $X_1,X_2$ and $A_1,A_2$. Taking the measurement of
gender bias as an example, word sets about science and art
can be used as the two target sets, masculine and feminine
names as the two attribute sets. Intuitively, the
null hypothesis means science and art are equally similar to
each masculine and feminine name. In the prior literature
it has been argued that if the null hypothesis cannot be
rejected, there is no significant gender bias. \enote{pd}{we
  should critisize this reasoning. The null and alternative
  hypothesis should be swapped. Has other work criticized
  this setup? Maybe we can do the test in addition in an
  alternative way?} The WEAT test statistic is defined as
\begin{eqnarray}
s(X_1,X_2,A_1,A_2)=\sum_{x\in X_1}s(x,A_1,A_2)\nonumber\\
-\sum_{x\in X_2}s(x,A_1,A_2),\nonumber
\end{eqnarray}
where
\begin{eqnarray}
s(x,A_1,A_2)=\mbox{mean}_{a\in A_1}cos(\vec{x},\vec{a})\nonumber\\
-\mbox{mean}_{a\in A_2}cos(\vec{x},\vec{a})\nonumber
\end{eqnarray}
$cos(\vec{x},\vec{a})$ denotes the cosine similarity between embedding vector $\vec{x}$ and $\vec{a}$. Intuitively, $s(x,A_1,A_2)$ measures the association of a word with the attributes, so the test statistic measures the differential association of the two target sets with the attributes. 

Let $\{({X_1}_i,{X_2}_i)\}_{i}$ denote all the partitions of $X_1\cup X_2$. The one-sided $p$-value of the permutation test is defined as $$Pr_i[s({X_1}_i,{X_2}_i,A_1,A_2)]>s(X_1,X_2,A_1,A_2)$$
\enote{pd}{I do not understand the notation fully. Is the p-value computed with respect to a single partition $i$?}
The effect size $d$-value is a normalized measure of how separated the two distributions of associations between the target and attribute are. It is defined as
\begin{eqnarray}
d=\frac{s(X_1,X_2,A_1,A_2)}{\mbox{std}_{x\in X_1 \cap X_2}s(x,A_1,A_2)}.\nonumber
\end{eqnarray}

\enote{hs}{above: $x\in X_1 \cap X_2$ or 
  $x\in X_1 \cup X_2$}

\enote{hs}{i think there is a summary sentence missing here:
  how do we use this to evaluate / compare debiasing methods?}

\subsection{Debiasing Methods}
 Many methods to remove gender bias have been proposed. The
 most common way is to define a gender direction (or, more
 generally, a subspace) by a set of gendered words, and
 debias the word embeddings in a post-processing
 projection. \citet{bolukbasi2016man} propose (i) \emph{hard
   debiasing}: they use the gendered words to compute the
 difference embedding vector as the gender direction; and
 (ii) \emph{soft debiasing},
 a
 machine learning based method
that combines
 the inner-products objective of word embedding and an
 objective to project the word embedding into an orthogonal
 gender subspace. Hard debiasing has been found to work
 better. \enote{pd}{should we mention hard-debiasing by mu
   et al here and explain the difference to bolukbasi?}
 \citet{dev2019attenuating} explored partial projection and
 some simple tricks to improve the hard debiasing
 method. \citet{zhao2019gender} applied the data
 augmentation and debiasing method of
 \citet{bolukbasi2016man} to mitigate gender bias on ELMo
 \citep{Peters:2018}. \citet{karve2019conceptor} introduce
 the debiasing conceptor: they shrine each
 principal component of the covariance matrix of the word
 embeddings to achieve a soft debiasing. Besides the above
 post-processing methods, \citep{zhao2018learning} propose
 GN-Glove: it debiases during training to learn word
 embeddings with protected attributes. The method we use
 here, DensRay, is similar to
hard debiasing in that we find
and eliminate a gender subspace in post-processing.
But DensRay can be solved efficiently in closed form and it
is more stable than hard debiasing.

\enote{hs}{above: what does ``shrine'' mean?}
 
\subsection{DensRay}
DensRay is an analytical method for identifying the
embedding subspace of certain linguistic features. Similar to the
methods mentioned in the previous section, we aim to
identify the ``gender subspace'' using a set of gendered words
$V:=\{v_1,v_2,\dots,v_n\}$ and their embeddings $E \in
R^{n\times d}$, thus for word $v_i$ we have the
corresponding embedding vector $e_{v_i}$. We
introduce a function $l$ for the gender attribute:
$l:V\to \{-1,1\}$;
e.g. $l(father)=1$, $l(sister)=-1$. The objective of DensRay
is to find an orthogonal matrix $Q\in R^{d\times d}$ such
that $EQ$ is gender-interpretable, specifically, the first
$k$ dimensions can be interpreted as the gender subspace.

Let $L_{=}:=\{(v,w)\in V\times V|l(v)=l(w)\}$ and define
$L_{\neq}$ analogously.  The DensRay objective
in \eqref{densray1} is to maximize the distance of the word
pairs from the same gender group ($L_{=}$) and minimize the
distance of the word pairs from the different gender group
($L_{\neq}$).
\begin{eqnarray}
    \max\limits_{q} 
    \sum_{(v,w)\in L_{\neq}}\alpha_{\neq}||q^Td_{vw}||^2_2\nonumber\\
    -\sum_{(v,w)\in L_{=}}\alpha_{=}||q^Td_{vw}||^2_2
\eqlabel{densray1}
\end{eqnarray}
where we define $d_{vw}:=e_v-e_w$. We also have $q\in R^d$
and $q^Tq=1$ since $Q$ is \enote{hs}{orthogonal or
  orthonormal?} orthogonal. $\alpha_{\neq},\alpha_{=}\in [0,1]$ are hyperparameters. Observing that $||x||^2_2=x^Tx$, objective \eqref{eq:densray1} can be simplified to:
\begin{eqnarray}
    \max \limits_{q} q^T(
    \sum_{(v,w)\in L_{\neq}}\alpha_{\neq}||d_{vw}d_{vw}^T||^2_2\nonumber\\
    \sum_{(v,w)\in L_{=}}\alpha_{=}||d_{vw}d_{vw}^T||^2_2)q\nonumber\\
    =:\max\limits_{q} q^TAq
\label{eq:densray2}
\end{eqnarray}

The objective in \eqref{eq:densray2} is maximizing the Rayleigh quotient of $A$ and $q$. Since $A$ is symmetric, we can get an analytical solution $q$ by the eigenvector with the max eigenvalue of $A$ \citep{horn1990matrix}. Thus the matrix of $k$ eigenvectors of $A$ ordered by the corresponding eigenvalues yields the matrix $Q$.



